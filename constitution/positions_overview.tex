\section{Positions}
\label{sec:positions}

Table~\ref{tab:positions} provides a summary of the administrative positions within CS Women. We expect the size of the eligible membership to fluctuate over time. When membership is small, the responsibilities of the non-necessary positions may be fulfilled by individuals in the necessary positions. When membership is large, the non-necessary positions offload some responsibilities from the necessary ones.

\begin{table}[h]
\centering
\begin{tabular}{l c c l c c}
	Position Title & Min Slots & Max Slots & Eligibility & Necessary & Description\\
	\hline
	Co-chair & 2 & 4 & U, G & X & \S\ref{sec:cochairs}\\
	Treasurer & 1 & 2 & G & X & \S\ref{sec:treasurer}\\
	Faculty Liaison & 1 & 1 & F & X & \S\ref{sec:facultyliaison} \\
	\gwis{} Liaison & 1 & 4 & G & X & \S\ref{sec:gwisliaison}\\
	Faculty Representative & 1 & 1 & G & & \S\ref{sec:facultyrep} \\
	Outreach Chair & 1 & 2 & U, G, P, S, O & & \S\ref{sec:outreach}\\
	Social Media Chair & 1 & 2 & U, G, P, F, S, O & & \S\ref{sec:socialmedia}\\
	Scribe & 1 & 2 & U, G, P, F, S, O &  & \S\ref{sec:scribe}\\
	Social Chair & 1 & 1 & U & & \S\ref{sec:social}
\end{tabular}
\caption{Summary of CS Women positions (\textbf{U}: Undergraduate student, \textbf{G}: Graduate student, \textbf{P}: Post-doctoral researcher, \textbf{F}: Faculty, \textbf{S}: Staff, \textbf{O}: Other).}
\label{tab:positions}
\end{table}