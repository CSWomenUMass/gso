\subsection{Co-Chairs}
\label{sec:cochairs}
Two graduate co-chairs and up to two undergraduate co-chairs provide the executive leadership and primary points of contact for the CS Women membership.

\subsubsection{Eligibility}
\label{sec:cochairs_eligibility}
Any CS Women general member who has completed at least two semesters of academic work in the College of Information and Computer Science (CICS) is eligible for the co-chair position. Graduate co-chairs must have either (a) held an assistantship for two semesters or (b) have completed 15 course credits applied to the graduate degree. Undergraduate co-chairs must have at least sophomore standing.

\subsubsection{Responsibilities}
\label{sec:cochairs_responsibilities}
The responsibilities of the co-chairs include but are not limited to:
\begin{enumerate}
	\item\label{item:lunch} Organizing the monthly lunches. The scheduling, format, content, and duration of the monthly lunches is determined primarily by the graduate co-chairs, with input from the undergraduate co-chairs when appropriate. See \S\ref{sec:obligations} for more information about the monthly lunches.
	\item\label{item:representation} Representing the organization at public events, upon request. Co-chairs may be asked to speak on behalf of the CS Women organization, or as representatives of women's interests in the CICS. Co-chairs should split responsibilities between graduate and undergraduate co-chairs, as appropriate. 
	\item\label{item:connection} Connecting the general membership with the resources they need. Members of the CS Women general membership or outside the community may seek academic, personal, or professional assistance from the co-chairs. Co-chairs should connect members with the appropriate resources, with input from the faculty liaison when in doubt. Forms of assistance include, but are not limited to:
	\begin{enumerate}
	\item\label{item:tutoring} Past co-chairs have received tutoring requests from UMass undergraduates, 5-college students, and individuals who are not members of the local academic community. These emails may be sent to the  graduate co-chairs only. Co-chairs are under no obligation to tutor students; however, they may do so if they choose. 
	\item\label{item:geo} Conflicts with a student's advisor. The co-chairs should connect the student with the current Graduate Employee Organization (GEO) stewards and/or the Graduate Program Director (GPD).
	\item\label{item:womens} Co-chairs may be asked to advise or advocate on behalf of women's issues in CICS. Co-chairs should consult with the elected Faculty Representative (\S\ref{sec:facultyrep}) and the appointed Faculty Liaison (\S\ref{sec:facultyliaison}).
	\item\label{item:psych} Students may reach out to graduate co-chairs when in need of emotional support or psychological services. As co-chairs are not required to undergo any training for counseling, \emph{they must encourage the student to talk to a professional}. Co-chairs should use their best judgement when 
	\end{enumerate}
	\item\label{item:review} Selecting new co-chairs and 
\end{enumerate}

\subsubsection{Term Length}
\label{sec:cochairs_termlength}

\subsubsection{Selection}
\label{sec:cochairs_selection}