\subsection{Co-Chairs}
\label{sec:cochairs}
Two graduate co-chairs and up to two undergraduate co-chairs provide the executive leadership and primary points of contact for the CS Women membership.

\subsubsection{Eligibility}
\label{sec:cochairs_eligibility}
Any CS Women general member who has completed at least two semesters of academic work in the College of Information and Computer Science (CICS) is eligible for the co-chair position. \textbf{Graduate co-chairs} must be enrolled in a graduate degree program in CICS, be in good academic standing, have attended at least three (3) CS Women sponsored events, and must have either (a) held an assistantship for two semesters or (b) have completed 15 course credits applied to the graduate degree. \textbf{Undergraduate co-chairs} must be enrolled in a CICS major or minor, must have at least sophomore standing and must be in good academic standing.

\subsubsection{Responsibilities}
\label{sec:cochairs_responsibilities}
The responsibilities of the co-chairs include but are not limited to:
\begin{enumerate}
	\item\label{item:lunch} Organizing the monthly lunches. The scheduling, format, content, and duration of the monthly lunches is determined primarily by the graduate co-chairs, with input from the undergraduate co-chairs when appropriate. See \S\ref{sec:obligations} for more information about the monthly lunches.
	\item\label{item:representation} Representing the organization at public events, upon request. Co-chairs may be asked to speak on behalf of the CS Women organization, or as representatives of women's interests in the CICS. Co-chairs should split responsibilities between graduate and undergraduate co-chairs, as appropriate. 
	\item\label{item:connection} Connecting the general membership with the resources they need. Members of the CS Women general membership or outside the community may seek academic, personal, or professional assistance from the co-chairs. Co-chairs should connect members with the appropriate resources, with input from the faculty liaison when in doubt. Forms of assistance include, but are not limited to:
	\begin{enumerate}
	\item\label{item:tutoring} Past co-chairs have received tutoring requests from UMass undergraduates, 5-college students, and individuals who are not members of the local academic community. These emails may be sent to the  graduate co-chairs only. Co-chairs are under no obligation to tutor students; however, they may do so if they choose. 
	\item\label{item:geo} Conflicts with a student's advisor. The co-chairs should connect the student with the current Graduate Employee Organization (GEO) stewards and/or the Graduate Program Director (GPD).
	\item\label{item:womens} Co-chairs may be asked to advise or advocate on behalf of women's issues in CICS. Co-chairs should consult with the elected Faculty Representative (\S\ref{sec:facultyrep}) and the appointed Faculty Liaison (\S\ref{sec:facultyliaison}).
	\item\label{item:psych} Students may reach out to graduate co-chairs when in need of emotional support or psychological services. As co-chairs are not required to undergo any training for counseling, \emph{they must encourage the student to talk to a professional}. Co-chairs should use their best judgement when determining whom to talk to for assistance.
	\end{enumerate}
	\item\label{item:review} Selecting new co-chairs and reviewing applications for other positions. New co-chairs are recruited and appointed by existing co-chairs. See \S\ref{sec:cochairs_selection} for more detail. Other positions, such as Social Media Chair~\S\ref{sec:socialmedia_selection}, Scribe~\S\ref{sec:scribe_selection}, and Social Chair~\S\ref{sec:socialchair_selection} are by application.
	\item\label{item:other} Graduate co-chairs should take on the responsibilities of the non-essential positions when those positions are not filled.
\end{enumerate}

\subsubsection{Term Length}
\label{sec:cochairs_termlength}
\paragraph{Graduate co-chairs.} Graduate co-chairs each serve one term corresponding to one calendar year. Terms begin January 1 and June 1. Unless a co-chair resigns from their position, terms should be staggered between the two co-chairs. A graduate student may not serve more than one term as a co-chair. 

\paragraph{Undergraduate co-chairs.} There are no restrictions on term length for undergraduate co-chairs. Undergraduate co-chairs may serve so long as they remain students at UMass. Undergraduate co-chairs who continue at UMass for their graduate studies have the same eligibility requirements for a graduate co-chair position as any other graduate student.

\subsubsection{Selection}
\label{sec:cochairs_selection}
Current co-chairs are responsible for recruiting their replacements. For graduate co-chairs, preference will be given to graduate students who have assistantships in CICS over the summer months.