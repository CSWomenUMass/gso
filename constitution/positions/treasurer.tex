\subsection{Treasurer}
\label{sec:treasurer}

\subsubsection{Responsibilities}
\label{sec:treasurer_responsibilities}
Once selected, treasurers must undergo GSO treasurer training.

\subsubsection{Eligibility}
\label{sec:treasurer_eligibility}
The treasurer position is open to any graduate member in good standing.

\subsubsection{Selection}
\label{sec:treasurer_selection}
When a treasurer position is open, the co-chairs solicit applications from the eligible members (See \S\ref{sec:treasurer_eligibility}) of the general membership. Applications are reviewed by the co-chairs and the faculty liaison. Selection should be announced within six (6) weeks of the call for applications. 

\subsubsection{Term Length}
\label{sec:treasurer_responsibilities}
\emma{Find out how often treasurer training is run. If it's run every semester, have term length be three semesters, accounting for a one-semester overlap to hand off treasurer responsibilities.}
Treasurers serve terms of two semesters. They may serve a maximum of two terms. The treasurer may be relieved from their responsibilities under the following circumstances:
\begin{itemize}
	\item Failure to fulfill responsibilities.
	\item Misappropriation of funds.
\end{itemize}
The following groups may relieve the treasurer from their position:
\begin{itemize}
	\item The faculty liaison, with approval of at least one co-chair.
	\item Unanimous approval of the co-chairs.
	\item Majority vote in the membership.
\end{itemize}
In all cases, the reason for the treasurer's removal must be put into writing.

\subsubsection{Approximate Time Commitment}
\emma{Find out how much time for training and doing budget stuff. My current estimate is that it's about half the time commitment of co-chairs.}
