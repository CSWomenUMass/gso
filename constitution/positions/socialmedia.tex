\subsection{Social Media Chair}
\label{sec:socialmedia}
The role of the social media chair is to publicize events and news related to CS Women. 

\subsubsection{Responsibilities}
\label{sec:socialmedia_responsibilities}
\paragraph{Social Media Outlets.} The social media chair should maintain an active presence on the following outlets:
\begin{itemize}
	\item \textbf{Facebook.} CS Women maintains a closed Facebook group, where students post articles of interest and events. The social media chair should publicize events here. Request administrative access from the current co-chairs.
		\paragraph{Group membership.} The social media chair is also responsible for inviting members to the Facebook group and approving individuals who have requested membership. The social media chair should send an email to the general membership publicizing the Facebook group at the beginning of the Fall term. When handling requests to join the group, any 5-College-affiliated individual should be approved. If the social media chair is not sure whether someone has 5-college affiliation, contact that person. 
		\paragraph{Posts.} Events, relevant news articles, and internal news should all be advertised in the group. Students may also use the Facebook group to find roommates, advertise personal interests, and generally promote an online social community.
	\item \textbf{Twitter.} CS Women maintains a Twitter account that should be used for group news. Re-tweet relevant news from the CICS account. The Twitter account is more actively followed by professional organizations and corporate affiliates. Tweet corporate-sponsored events with the sponsor's hashtag(s). Also tweet professional updates from the membership with the hashtag \verb|#goodnews|. Request the password from the co-chairs.
	\item \textbf{Github.} All CS Women documentation is kept on Github. The social media chair will be tagged in relevant posts.
\end{itemize}
Although not required, the social media chair is also encouraged to advertise events offline as well. This can be done through postering in the Computer Science building and other appropriate locations.

Note that the CS Women leadership often receives unsolicited recruitment requests. Requests for participation in hackathons are an increasingly popular recruitment mechanism, as are ``free'' schools/camps/etc. CS Women has corporate affiliates via CICS, who are given priority in recruitment. Any requests from parties that are not affiliates need to be redirected to the Treasurer, who will put them in touch with the appropriate liaison in CICS.

\label{sec:socialmedia_eligibility}
\paragraph{Events} Events are advertised on both Facebook and Twitter. The events we expect the social media chair to publicize include but are not limited to:
\begin{itemize}
	\item CS Women lunches. These are typically held monthly during the academic year. See \S\ref{sec:lunches}.
	\item Outreach events. Coordinate with the Outreach Chair (\S\ref{sec:outreach}), if that position is filled; otherwise, coordinate with the Co-chairs (\S\ref{sec:cochairs}).
	\item Female technical speakers. The social media chair should be subscribed to \verb|seminars@cs.umass.edu| and should make contact with the organizers of the following events:
		\begin{itemize}
		\item Machine Learning and Friends Lunch (MLFL).
		\item Systems Lunch
		\item Computational Social Science Seminar (CSS Lunch).
		\item Distinguished Lecturer Series
		\end{itemize}
		The organizers of these events rotate; refer to the department wiki for the latest news.
	\item Undergraduate events. The undergraduate co-chairs may organize events after hours in the CS Building. These events may be open to individuals outside the CS Women membership. Undergraduate events are best advertised... \cassie{What's the best way to reach undergrads?}.
	\item General population events. Advertise any other events occurring in the Pioneer Valley that may be of interest to our membership.
\end{itemize}

\paragraph{News} Our news updates include both information about member's accomplishments and notifications of upcoming deadlines. We advertise professional updates from our membership on Twitter using the hashtag \verb|#goodnews|. \verb|#goodnews| updates include, but are not limited to:
	\begin{itemize}
	\item Fellowships and scholarships awarded to members.
	\item Paper acceptances.
	\item Awards.
	\item Receipt of prestigious roles in external organizations.
	\end{itemize}
	Deadlines for awards, scholarships, fellowships, and prestigious positions should also be tweeted one month, one week, and one day before the deadlines. Relevant deadlines include, but are not limited to:
\begin{itemize}
	\item Grace Hopper Celebration Scholarship.
	\item Graduate Cohort Application Deadline.
	\item Academic scholarships targeted toward women (e.g., Microsoft women's scholarship, Anita Borg scholarship, etc.)
	\item ACM-W events.
\end{itemize}


\subsubsection{Eligibility}
\label{sec:socialmedia_eligibility}
Any interested individual in the general membership is eligible. 

\subsubsection{Selection}
\label{sec:socialmedia_selection}
Current co-chairs are responsible for advertising the position to the general membership when it becomes available. Up to two social media chairs will be chosen. Preference will be given to distribute the two available positions between a graduate and an undergraduate member.


\subsubsection{Term Length}
\label{sec:socialmedia_termlength}
The term for the social media chair is one semester. There are no restrictions on the number of terms served.
\subsubsection{Approximate Time Commitment}
Expect to spend 1-2 hours per week on obligations. Since there are several accounts to manager, we recommend using a dedicated browser to handle all account information. For example, if you typically use Chrome for logging in to your personal accounts, log in to the CS Women accounts using Firefox. This should cut down on the overhead of managing multiple social media accounts.